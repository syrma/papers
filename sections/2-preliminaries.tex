\section{Preliminaries}
\label{sec:prelim}
This work builds on past theory on Reinforcement Learning and Policy Gradients. In this section, we will briefly summarize the essential structure of the RL problem and the categories of solutions relevant in the context of the present work.

\subsection{Markov Decision Processes}

Reinforcement Learning is the field of Machine Learning concerned with decision making within an uncertain environment, acting upon it through actions with uncertain consequences, with the aim of achieving a specific goal while maximizing a numerical reward. A distinguishing feature of RL is that the agent interacts with the environment and pieces together a policy $\pi_theta$––the solution to the problem relying purely on the feedback of the environment with no input from human experts in any form. 

It is possible to model virtually any problem that corresponds to the description above using a mathematical framework called a Markov Decision Process (MDP.) 

The MDP is defined by the tuple ($\mathcal{S}, \mathcal{A}, \mathcal{P}, r, \rho_0$) such that:
\begin{itemize}
\item $\mathcal{S}$ is the set of states, 
\item $\mathcal{A}$ is the set of actions, 
\item $\mathcal{P}: \mathcal{S}\times\mathcal{A}\times\mathcal{S}\mapsto \mathbb{R}$ is the transition probability distribution,
\item $r: \mathcal{S}\times\mathcal{A}\times\mathcal{S} \mapsto \mathbb{R}$ is the reward function,
\item $\rho_0:\mathcal{S}\mapsto\mathbb{R}$ is the initial state distribution.
\end{itemize}

\subsection{Policy Optimization}

%\[
%g = \mathbb{E}
%\left[\sum^{\infty}_{t=0} \Psi_t \nabla_{\theta}\log\pi_{\theta}(a_t | s_t) \right].
%\]
%Where the most popular choices for the weight $\Psi_t$ are the state_value function, 

\[
g = \mathbb{E}
\left[\sum^{\infty}_{t=0} A^{\pi}(s_t, a_t) \nabla_{\theta}\log\pi_{\theta}(a_t | s_t) \right].
\]


Where $A^{\pi}(s_t, a_t)$ is the advantage function, which indicates for each action $a_t$ whether it is better or worse than the average action generated by the policy $\pi_{\theta}$ in a state $s_t$. It is formally defined as:

\begin{equation}
A^\pi(s_t,a_t) = Q^\pi(s_t, a_t) - V^\pi(s_t)
\end{equation}
for
\begin{align*}
&V(s_t) = \mathbb{E}\left[\sum_{l=0}^{\infty} r_l | s_0 = s_t \right],\\
&Q(s_t, a_t) = \mathbb{E}\left[\sum_{l=0}^{\infty} r_l | s_0 = s_t, a_0 = a_t \right]
\end{align*}

Though in practice, the advantage function is not known and is often replaced by an estimator $\hat{A}$ following one of several known estimation methods.

\subsection{Proximal Policy Optimization}

PPO (Proximal Policy Optimization\cite{schulman2017proximal}) is one of the most popular Deep RL algorithms, which has since been applied to a wide range of domains from robotics\cite{andrychowicz2020learning} to chip design\cite{mirhoseini2021graph}. It is a comparatively accessible algorithm to implement and deploy. A trend in model-free RL has been to improve performance through increasingly complex methods, which comes at the great cost of making RL methods hard to reproduce and validate as well as unpractical to use in physical systems. This conflicts with some of the stated goals of Reinforcement Learning research, namely the aim to create a general paradigm that requires less human intervention than any other conventional Artificial Intelligence paradigm. This makes PPO

\[
\theta_{k+1} = \arg \max_{\theta} \frac{1}{|\mathcal{D}_k|T}\sum_{t}
\] 


\subsection{Actor-critic style RL}


