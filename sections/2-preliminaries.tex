\section{Preliminaries}
\label{sec:prelim}
This work builds on past theory on Reinforcement Learning and Policy Gradients. In this section, we will briefly summarize the essential structure of the RL problem and the categories of solutions relevant to the context of the present work.

\subsection{Markov Decision Processes}

Reinforcement Learning is the field of Machine Learning concerned with decision making within an uncertain environment, acting upon it through actions with uncertain consequences, with the aim of achieving a specific goal while maximizing a numerical reward. A distinguishing feature of RL is that the agent interacts with the environment and pieces together a policy $\pi_theta$—the solution to the problem relying purely on the feedback of the environment with no input from human experts in any form. 

It is possible to model virtually any problem that corresponds to the description above using a mathematical framework called a Markov Decision Process (MDP.) 

The MDP is defined by the tuple ($\mathcal{S}, \mathcal{A}, \mathcal{P}, r, \rho_0$) where $\mathcal{S}$ is the set of states, $\mathcal{A}$ is the set of actions, $\mathcal{P}: \mathcal{S}\times\mathcal{A}\times\mathcal{S}\mapsto \mathbb{R}$ is the transition probability distribution, $r: \mathcal{S}\times\mathcal{A}\times\mathcal{S} \mapsto \mathbb{R}$ is the reward function, and $\rho_0:\mathcal{S}\mapsto\mathbb{R}$ is the initial state distribution.


\subsection{Proximal Policy Optimization}

PPO (Proximal Policy Optimization\cite{schulman2017proximal}) is one of the most popular Deep RL algorithms, which has since been applied to a wide range of domains from robotics\cite{andrychowicz2020learning} to chip design\cite{mirhoseini2021graph}. It is a comparatively accessible algorithm to implement and deploy. A trend in model-free RL has been to improve performance through increasingly complex methods, which comes at the great cost of making RL methods hard to reproduce and validate as well as unpractical to use in physical systems. This is at odds with the very purpose of advancing Reinforcement Learning research, a field aiming to create a general paradigm that requires less human intervention than any other known Artificial Intelligence model. 
